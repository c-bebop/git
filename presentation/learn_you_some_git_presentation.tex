\documentclass[aspectratio=169]{beamer}

% This document was written with the help of
% http://en.wikibooks.org/wiki/LaTeX/Presentations

% The theme of this presentation
\usetheme{default}

\definecolor{blue}{HTML}{196284}

\setbeamercolor{title}{fg=blue}
\setbeamercolor{titlelike}{fg=blue}

% Makes more space between new lines
\setlength{\parskip}{10pt plus 1pt minus 1pt}

% Open Sans Package
\usepackage[default,scale=0.95]{opensans}
\usepackage[T1]{fontenc}



% Title page
\title[Learn You some GIT] % (optional, only for long titles)
{Learn You some GIT}
\author[Florian Willich]{Florian Willich}
\institute[BIT]
{
  Quality and Usability Lab\\
  Berlin Institute of Technology
}
\date{\today}

\subject{Computer Science}

\begin{document}

\frame{\titlepage}

\begin{frame}
\frametitle{Table of Contents}
\tableofcontents[currentsection]
\end{frame}

\begin{frame}
\frametitle{Create Repositories}

\$ \textcolor{blue}{git init} project-name\\
Creates a new local repository with the specified name

\$ \textcolor{blue}{git clone} url\\
Downloads a project and its entire version history but only the master branch!

\$ \textcolor{blue}{git fetch} remote-branch/local-branch\\
lets you fetch the remote branch and create a local branch

\$ \textcolor{blue}{git checkout} local-branch\\
switch to the branch\\

\end{frame}

\begin{frame}
  
\frametitle{Make Changes}

\$ \textcolor{blue}{git status} project-name\\
Most important command! Lists all new or modified files to be committed

\$ \textcolor{blue}{git add} file\\
Snapshots the file in preparation for versioning

\$ \textcolor{blue}{git commit -m} "descriptive message"\\
Records file snapshots permanently in version history

\$ \textcolor{blue}{git commit -am} "descriptive message"\\
Snapshots all tracked files in preparation for versioning \& records file snapshots permanently in version history

\end{frame}

\begin{frame}

\frametitle{Group Changes}

\$ \textcolor{blue}{git branch}\\
Lists all local branches in the current repository

\$ \textcolor{blue}{git branch} branch-name\\
Creates a new branch with the specified branch name

\$ \textcolor{blue}{git checkout} branch-name\\
Switches to the specified branch and updates the working directory

\$ \textcolor{blue}{git merge} branch-name\\
Combines the specified branch's history into the current branch

\$ \textcolor{blue}{git branch -d} branch-name\\
Deletes the specified branch

\end{frame}

\begin{frame}

\frametitle{Suppress Tracking}

By creating a file called \textcolor{blue}{.gitignore} (yes its a hidden file) in the root directory, you can specify all the files you want git to ignore.

Examples for files you don't want to track:
\begin{itemize}

\item *.log
\item *.config
\item my-secret-passwords.secret
\item Any IDE related files

\end{itemize}

\begin{frame}

\frametitle{Save Fragments}



\end{frame}


% etc
\end{document}